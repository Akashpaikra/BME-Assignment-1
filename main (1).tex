\documentclass[12pt]{article}
\usepackage[english]{babel}
\usepackage{natbib}
\usepackage{url}
\usepackage[utf8x]{inputenc}
\usepackage{amsmath}
\usepackage{graphicx}
\graphicspath{{images/}}
\usepackage{parskip}
\usepackage{fancyhdr}
\usepackage{vmargin}
\setmarginsrb{3 cm}{2.5 cm}{3 cm}{2.5 cm}{1 cm}{1.5 cm}{1 cm}{1.5 cm}

\title{MEDICAL DEVICES}					
\author{21111004}								
\date{28 JAN 2022}						
\makeatletter
\let\thetitle\@title
\let\theauthor\@author
\let\thedate\@date
\makeatother

\pagestyle{fancy}
\fancyhf{}
\rhead{\theauthor}
\lhead{\thetitle}
\cfoot{\thepage}

\begin{document}
\begin{titlepage}
	\centering
    \includegraphics[scale = 0.20]{logo.jpg}\\[1.0 cm]	
    \textsc{\LARGE National Institute Of Technology \newline\\\\ RAIPUR}\\[2.0 CM]
    
	\textsc{\Large ASSIGNMENT 01}\\[0.5 cm]				% Course Code
	\rule{\linewidth}{0.4 mm} \\[0.4 cm]
	{ \huge \bfseries \thetitle}\\
	\rule{\linewidth}{0.4 mm} \\[1.5 cm]
	
	\begin{minipage}{0.6\textwidth}
		\begin{flushleft} \large
			\emph{Submitted To:}\\
			Saurabh Gupta\\
            Department Of Basic Biomedical Engineering\\
			\end{flushleft}
			\end{minipage}~
			\begin{minipage}{0.4\textwidth}
            
			\begin{flushright} \large
			\emph{Submitted By :}\\
			Akash Paikra\\
            21111006\\
		\end{flushright}
        
	\end{minipage}\\[2 cm]
\end{titlepage}

\tableofcontents
\pagebreak







\section{POLYMERASE CHAIN REACTION(PCR)}
Polymerase chain reaction (PCR) is a method widely used to rapidly make millions to billions of copies (complete copies or partial copies) of a specific DNA sample, allowing scientists to take a very small sample of DNA and amplify it (or a part of it) to a large enough amount to study in detail. PCR was invented in 1983 by the American biochemist Kary Mullis at Cetus Corporation. It is fundamental to many of the procedures used in genetic testing and research, including analysis of ancient samples of DNA and identification of infectious agents. Using PCR, copies of very small amounts of DNA sequences are exponentially amplified in a series of cycles of temperature changes. PCR is now a common and often indispensable technique used in medical laboratory research for a broad variety of applications including biomedical research and criminal forensics.



\begin{figure}[h]
   \centering
   \includegraphics[scale=0.4]{pcr}
   \caption{polymerase chain reaction}
   \label{fig_pcr}
\end{figure}


The majority of PCR methods rely on thermal cycling. Thermal cycling exposes reactants to repeated cycles of heating and cooling to permit different temperature-dependent reactions—specifically, DNA melting and enzyme-driven DNA replication. PCR employs two main reagents—primers (which are short single strand DNA fragments known as oligonucleotides that are a complementary sequence to the target DNA region) and a DNA polymerase. In the first step of PCR, the two strands of the DNA double helix are physically separated at a high temperature in a process called nucleic acid denaturation. In the second step, the temperature is lowered and the primers bind to the complementary sequences of DNA. The two DNA strands then become templates for DNA polymerase to enzymatically assemble a new DNA strand from free nucleotides, the building blocks of DNA. As PCR progresses, the DNA generated is itself used as a template for replication, setting in motion a chain reaction in which the original DNA template is exponentially amplified.


Almost all PCR applications employ a heat-stable DNA polymerase, such as Taq polymerase, an enzyme originally isolated from the thermophilic bacterium Thermus aquaticus. If the polymerase used was heat-susceptible, it would denature under the high temperatures of the denaturation step. Before the use of Taq polymerase, DNA polymerase had to be manually added every cycle, which was a tedious and costly process.


Applications of the technique include DNA cloning for sequencing, gene cloning and manipulation, gene mutagenesis; construction of DNA-based phylogenies, or functional analysis of genes; diagnosis and monitoring of genetic disorders; amplification of ancient DNA; analysis of genetic fingerprints for DNA profiling (for example, in forensic science and parentage testing); and detection of pathogens in nucleic acid tests for the diagnosis of infectious diseases.



\subsection{Uses of PCR}
Prospective parents can be tested for being genetic carriers, or their children might be tested for actually being affected by a disease. DNA samples for prenatal testing can be obtained by amniocentesis, chorionic villus sampling, or even by the analysis of rare fetal cells circulating in the mother's bloodstream. PCR analysis is also essential to preimplantation genetic diagnosis, where individual cells of a developing embryo are tested for mutations.

                PCR can also be used as part of a sensitive test for tissue typing, vital to organ transplantation. As of 2008, there is even a proposal to replace the traditional antibody-based tests for blood type with PCR-based tests.
Many forms of cancer involve alterations to oncogenes. By using PCR-based tests to study these mutations, therapy regimens can sometimes be individually customized to a patient. PCR permits early diagnosis of malignant diseases such as leukemia and lymphomas, which is currently the highest-developed in cancer research and is already being used routinely. PCR assays can be performed directly on genomic DNA samples to detect translocation-specific malignant cells at a sensitivity that is at least 10,000 fold higher than that of other methods. PCR is very useful in the medical field since it allows for the isolation and amplification of tumor suppressors. Quantitative PCR for example, can be used to quantify and analyze single cells, as well as recognize DNA, mRNA and protein confirmations and combinations.
\clearpage





\section{TRADMILL}

A treadmill is a device generally used for walking, running, or climbing while staying in the same place. Treadmills were introduced before the development of powered machines to harness the power of animals or humans to do work, often a type of mill operated by a person or animal treading the steps of a treadwheel to grind grain. In later times, treadmills were used as punishment devices for people sentenced to hard labor in prisons. The terms treadmill and treadwheel were used interchangeably for the power and punishment mechanisms.

More recently, treadmills have instead been used as exercise machines for running or walking in one place. Rather than the user powering a mill, the device provides a moving platform with a wide conveyor belt driven by an electric motor or a flywheel. The belt moves to the rear, requiring the user to walk or run at a speed matching the belt. The rate at which the belt moves is the rate of walking or running. Thus, the speed of running may be controlled and measured. The more expensive, heavy-duty versions are motor-driven (usually by an electric motor). The simpler, lighter, and less expensive versions passively resist the motion, moving only when walkers push the belt with their feet. The latter are known as manual treadmills.

Treadmills continue to be the biggest selling exercise equipment category by a large margin. As a result, the treadmill industry has hundreds of manufacturers throughout the world.

\begin{figure}[h]
   \centering
   \includegraphics[scale=0.1]{treadmill}
   \caption{TREADMILL}
   \label{fig_treadmill}
\end{figure}


\subsection{Treadmills for power}
Treadmills as power sources originated in antiquity.These ancient machines had three major types of design.[4] The first was a horizontal bar jutting out of a vertical shaft. It rotated around a vertical axis, driven by an ox or other animal walking in a circle and pushing the bar. Humans were also used to power these. The second design was a vertical wheel, a treadwheel, that was powered by climbing in place instead of walking in circles. This is similar to what we know today as the hamster wheel. The third design also required climbing but used a sloped, moving platform instead.

Treadmills as muscle powered engines originated roughly 4000 years ago.[citation needed] Their primary use was to lift buckets of water. This same technology was later adapted to create rotary grain mills and the treadwheel crane. It was also used to pump water and power dough-kneading machines and bellows.


\subsection{Treadmills for exercise }
The forerunner of the exercise treadmill was designed to diagnose heart and lung diseases, and was invented by Robert Bruce and Wayne Quinton at the University of Washington in 1952.Kenneth H. Cooper's research on the benefits of aerobic exercise, published in 1968, provided a medical argument to support the commercial development of the home treadmill and exercise bike.          Among users of treadmills today are medical facilities (hospitals, rehabilitation centers, medical and physiotherapy clinics, institutes of higher education), sports clubs, biomechanics institutes, orthopedic shoe shops, running shops, Olympic training centers, universities, fire-training centers, NASA, test facilities, police forces and armies, gyms and even home user. Medical treadmills are class IIb active therapeutic devices and also active devices for diagnosis. With their very powerful (e.g. 3.3 kW = 4.5 HP) electric motor powered drive system, treadmills deliver mechanical energy to the human body through the moving running belt of the treadmill. The subject does not change their horizontal position and is passively moved and forced to catch up with the running belt underneath their feet. The subject can also be attached in a safety harness, unweighting system, various supports or even fixed in and moved with a robotic orthotic system utilizing the treadmill.
\clearpage
             
             
             
             
      
\section{MUSCLE STIMULATOR }
Medical treadmills are class IIb active therapeutic devices and also active devices for diagnosis. With their very powerful (e.g. 3.3 kW = 4.5 HP) electric motor powered drive system, treadmills deliver mechanical energy to the human body through the moving running belt of the treadmill. The subject does not change their horizontal position and is passively moved and forced to catch up with the running belt underneath their feet. The subject can also be attached in a safety harness, unweighting system, various supports or even fixed in and moved with a robotic orthotic system utilizing the treadmill.




\begin{figure}
    \centering
    \includegraphics[scale=1]{elec}
    \caption{MUSCLE STIMULATOR}
    \label{fig_elec.}
\end{figure}

\subsection{uses of electrical muscle stimulator }



\subsection{physical rehabilitation}
In medicine, EMS is used for rehabilitation purposes, for instance in physical therapy in the prevention muscle atrophy due to inactivity or neuromuscular imbalance, which can occur for example after musculoskeletal injuries (damage to bones, joints, muscles, ligaments and tendons). This is distinct from transcutaneous electrical nerve stimulation (TENS), in which an electric current is used for pain therapy. In the case of TENS, the current is usually sub-threshold, meaning that a muscle contraction is not observed.

For people who have progressive diseases such as cancer or chronic obstructive pulmonary disease, EMS is used to improve muscle weakness for those unable or unwilling to undertake whole-body exercise. EMS may lead to statistically significant improvement in quadriceps muscle strength, however, further research is needed as this evidence is graded as low certainity. The same study also indicates that EMS may lead to increased muscle mass. Low certainity evidence indicates that adding EMS to an existing exercise programme may help people who are unwell spend fewer days confined to their beds.

During EMS training, a set of complementary muscle groups (e.g., biceps and triceps) are often targeted in alternating fashion, for specific training goals, such as improving the ability to reach for an item.




\subsection{weight loss}



The FDA rejects certification of devices that claim weight reduction.[16] EMS devices cause a calorie burning that is marginal at best: calories are burnt in significant amount only when most of the body is involved in physical exercise: several muscles, the heart and the respiratory system are all engaged at once.[17] However, some authors imply that EMS can lead to exercise, since people toning their muscles with electrical stimulation are more likely afterwards to participate in sporting activities as the body becomes ready, fit, willing and able to take on physical activity.
\clearpage

\section{FULL AUTOMATED ANALYSER}

An automated analyser is a medical laboratory instrument designed to measure different chemicals and other characteristics in a number of biological samples quickly, with minimal human assistance. These measured properties of blood and other fluids may be useful in the diagnosis of disease.

Photometry is the most common method for testing the amount of a specific analyte in a sample. In this technique, the sample undergoes a reaction to produce a color change. Then, a photometer measures the absorbance of the sample to indirectly measure the concentration of analyte present in the sample. The use of an Ion Selective Electrode (ISE) is another common analytical method that specifically measures ion concentrations. This typically measures the concentrations of sodium, calcium or potassium present in the sample.

There are various methods of introducing samples into the analyser. Test tubes of samples are often loaded into racks. These racks can be inserted directly into some analysers or, in larger labs, moved along an automated track. More manual methods include inserting tubes directly into circular carousels that rotate to make the sample available. Some analysers require samples to be transferred to sample cups. However, the need to protect the health and safety of laboratory staff has prompted many manufacturers to develop analysers that feature closed tube sampling, preventing workers from direct exposure to samples.Samples can be processed singly, in batches, or continuously.

\begin{figure}[h]
    \centering
    \includegraphics[scale=0.5]{Auto}
    \caption{FULL AUTOMATED ANALYSER}
    \label{fig_Auto}
\end{figure}

\subsection{Routine biochemistry analyser}

These are machines that process a large portion of the samples going into a hospital or private medical laboratory. Automation of the testing process has reduced testing time for many analytes from days to minutes. The history of discrete sample analysis for the clinical laboratory began with the introduction of the "Robot Chemist" invented by Hans Baruch and introduced commercially in 1959.

The AutoAnalyzer is an early example of an automated chemistry analyzer using a special flow technique named "continuous flow analysis (CFA)", invented in 1957 by Leonard Skeggs, PhD and first made by the Technicon Corporation. The first applications were for clinical (medical) analysis. The AutoAnalyzer profoundly changed the character of the chemical testing laboratory by allowing significant increases in the numbers of samples that could be processed. Samples used in the analyzers include, but are not limited to, blood, serum, plasma, urine, cerebrospinal fluid, and other fluids from within the body. The design based on separating a continuously flowing stream with air bubbles largely reduced slow, clumsy, and error-prone manual methods of analysis. The types of tests include enzyme levels (such as many of the liver function tests), ion levels (e.g. sodium and potassium, and other tell-tale chemicals (such as glucose, serum albumin, or creatinine).

Simple ions are often measured with ion selective electrodes, which let one type of ion through, and measure voltage differences. Enzymes may be measured by the rate they change one coloured substance to another; in these tests, the results for enzymes are given as an activity, not as a concentration of the enzyme. Other tests use colorimetric changes to determine the concentration of the chemical in question. Turbidity may also be measured.
\clearpage


\section{ULTRASOUND DIAGNOSTIC}

Medical ultrasound includes diagnostic techniques (mainly imaging techniques) using ultrasound, as well as therapeutic applications of ultrasound. In diagnosis, it is used to create an image of internal body structures such as tendons, muscles, joints, blood vessels, and internal organs, to measure some characteristics (e.g. distances and velocities) or to generate an informative audible sound. Its aim is usually to find a source of disease or to exclude pathology. The usage of ultrasound to produce visual images for medicine is called medical ultrasonography or simply sonography. The practice of examining pregnant women using ultrasound is called obstetric ultrasonography, and was an early development of clinical ultrasonography.
          
          Compared to other medical imaging modalities, ultrasound has several advantages. It provides images in real-time, is portable, and can consequently be brought to the bedside. It is substantially lower in cost than other imaging strategies and does not use harmful ionizing radiation.[2] Drawbacks include various limits on its field of view, the need for patient cooperation, dependence on patient physique, difficulty imaging structures obscured by bone, air or gases,[note 1] and the necessity of a skilled operator, usually with professional training.


An ultrasound result on fetal biometry printed on a piece of paper.
Sonography (ultrasonography) is widely used in medicine. It is possible to perform both diagnosis and therapeutic procedures, using ultrasound to guide interventional procedures such as biopsies or to drain collections of fluid, which can be both diagnostic and therapeutic. Sonographers are medical professionals who perform scans which are traditionally interpreted by radiologists, physicians who specialize in the application and interpretation of medical imaging modalities, or by cardiologists in the case of cardiac ultrasonography (echocardiography). Increasingly, physicians and other healthcare professionals who provide direct patient care are using ultrasound in office and hospital practice (point-of-care ultrasound).[3]

Sonography is effective for imaging soft tissues of the body. Superficial structures such as muscle, tendon, testis, breast, thyroid and parathyroid glands, and the neonatal brain are imaged at higher frequencies (7–18 MHz), which provide better linear (axial) and horizontal (lateral) resolution. Deeper structures such as liver and kidney are imaged at lower frequencies (1–6 MHz) with lower axial and lateral resolution as a price of deeper tissue penetration


\begin{figure}[h]
   \centering
   \includegraphics[scale=1]{ultrasound}
   \caption{ULTRASOUND DIAGNOSTIC}
   \label{fig_ultrasound}
\end{figure}

\subsection{Anesthesiology}
In anesthesiology, ultrasound is commonly used to guide the placement of needles when injecting local anaesthetic solutions in the proximity of nerves identified within the ultrasound image (nerve block). It is also used for vascular access such as cannulation of large central veins and for difficult arterial cannulation. Transcranial Doppler is frequently used by neuro-anesthesiologists for obtaining information about flow-velocity in the basal cerebral vessels.







































\end{document}